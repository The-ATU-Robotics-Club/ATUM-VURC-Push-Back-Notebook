\chapter{About This Notebook}

\section{The Use of \LaTeX{}} This notebook was created using \LaTeX{}, a free,
open source typesetting program that is especially useful for technical
documentation .  We chose to use \LaTeX{} for the following
reasons:
\begin{itemize}
      \item \LaTeX{} provides greater control over the style and format of
            documents than traditional word processors. This is especially true
            for:
            \begin{itemize}
                  \item Long documents
                  \item Technical documents
            \end{itemize}
      \item Because \LaTeX{} documents are expressed with plain text, it
            provides an avenue for future automated documentation tools.
      \item Most practically, \LaTeX{} is a useful tool career-wise, so this
            notebook is a good opportunity to get practice with the software.
\end{itemize}
This notebooks format is inspired heavily by a template provided by Purdue
SIGBots  and an open source controls engineering
textbook by Tyler Veness  (which will be
referenced again).
\section{Formatting}
This notebook is presently divided into several parts which will now be
described in more detail.
\begin{itemize}
      \item \textit{Preface} deals with the notebook, team and club procedures,
            logistics, overall structure of our program, general game analysis,
            and design philosophy.
      \item \textit{Hardware} deals with the design process for the robots.
      \item \textit{Software} deals with the design process for software used
            with both robots as well as potentially custom electronics.
      \item \textit{Appendices} contain information that would not fit nicely in
            other sections of the notebook, such as commit logs, meeting notes,
            etcetera.
\end{itemize}
The sections regarding the hardware and software contain largely  "highlights"
of the team/club meetings. For a more traditional notebook experience, the
meeting note section of the appendix and the "logs" in each chapter should be
especially considered. This notebook's version history was recorded using git
and was distributed using GitHub. \textbf{To ensure that the notebook was made
      alongside the rest of the project, the commit history of the notebook repository
      is attached. Additionally, the following QR code leads to the notebook
      repository.}

\pagebreak
\begin{center}
      \
      \vfill
      \qrcode[height = .6\linewidth]{https://github.com/The-ATU-Robotics-Club/ATUM-VURC-Push-Back-Notebook}

      \vspace*{40pt}
      \url{https://github.com/The-ATU-Robotics-Club/ATUM-VURC-Push-Back-Notebook}

      \textit{The link above leads to the GitHub repository for this notebook, it is already public for this competition.}

      \textit{If there is any problems accessing the link, the commit history is also attached.}

      The PDF version may be preferred for reading, as the table of contents
      provides hyperlinks to various sections you will have to download the
      file. See the "releases" for the current PDF file.
      \vfill
\end{center}
\pagebreak

\addtocontents{toc}{\setcounter{tocdepth}{1}}

\chapter{Technology \& Methodology}
\section{git/GitHub}
Git is the version control software we use to manage versioning for any files and we use
GitHub to host our repositories. Tracked files include both software and CAD files, as well
as organizational documents such as our constitution. The use of GitHub became necessary
for CAD after the discontinuation of GrabCAD, which we used in previous years.

GitHub also provides several project management tools, such as GitHub Projects. A more detailed description of our usage of GitHub Projects can be found in the appendix,
but in summary it acts as our software for Gantt charts and Kanban boards. 


GitHub has been instrumental not only in tracking the progress of tasks and who is responsible for completing them, but is also employed for some automation. For instance, this notebook uses GitHub Actions to automatically add changes to our GitHub Project (effectively tracking our Gantt chart and Kanban board), add the commit logs for all of our repositories, adding general club meeting notes and the code repository to this notebook.
\newline

\section{Discord}
Three years ago, we created a discord server that we could use to communicate between team members. Within this server, we have various channels that we use to record day-to-day activities within our lab. It is how we announce team meetings, things that are needed to be done towards the robot, and how we keep record of the progress we make through taking pictures. Moreover, this is a quick and easy way to get a hold of team
members and make sure everyone stays up to date on what is happening throughout the semester. 

We additionally have several bots in the server to assist with certain functions related to the club. These bots include:
\begin{itemize}
\item Dyno: provides additional general discord commands.
\item vexibot: provides team, tournament, and skills information for VRC.
\item Compiler: compiles and runs provided snippets of code.
\item ATUM’s Family Bot: provides support for adding to the order form. The ATUM’s Family Bot in particular was custom-made for the club by a former member, Braden, four years ago.
\end{itemize}

With Push Back, we have updated the server to better support a longer lifetime, with support for visitors and archiving. 
\newline


\section{Club and Team Organization}
Our team consists of various members that all bring a different skill set to the table. We do
not discriminate on who is allowed to join by race, gender, sexual orientation, major, or any
other unique differences. Instead, we focus on building a positive environment that everyone
can feel comfortable working in as a team. Through building off each other’s individual skill
sets, we believe that we are able to create a productive robotics team. Within our team,
different members are allowed to pick a role they feel most comfortable with, however, that
does not withhold them from being allowed to help with any other role of the club.

If you would like to learn more about our club organization, the constitution of our club is attached in this notebook in the appendix. 
\newline 
\chapter{Members}


\chapter{Game Overview}
\label{chapter:Game Overview}

\subsection{Game Description}
VEXU Pusback is a University vs University robotics competition. In which both schools will compete with 2 robots in a 2:00 minute match consisting of a 30 second autonomous period, and a one minute and 30 second driver control period. 
