\setcounter{tocdepth}{1}
\chapter{Early Season}

\section{Design}
This section will discuss the proposed design solutions for the 2024-2025 Vex Push Back design challenge, highlighting the strengths and weaknesses of well-known mechanisms and prototypes. We will then propose designs that can be used for both robots.
\newline


\subsection{Design Philosophy}
As a VURC team, we have the ability to manufacture parts that are created using legal raw stock as defined by $<$VUR4$>$ in the game manual. See \ref{fig:VUR4} for $<$VUR4$>$. As the first designs of the season, we will choose not to design our own gears or sprockets initially. Having all the vex gears to choose from will allow us to save time and immediately jump into building any subsystems that require gears or sprockets. Moreover, we will use vex parts for the majority of the first robot designs as it will allow us to tinker with many prototypes of ideas without having to dedicate potential days of resources into something that could easily be replicated with vex parts that are readily available. However, this does not mean that we will not fabricate parts of our own.

\begin{figure} [h!]
    \centering
    \includegraphics[width=0.5\linewidth]{images/misc/VUR4.png}
    \caption{VUR4}
    \label{fig:VUR4}
\end{figure}

\subsection{Building Two 15" Robots}
This year, we have decided to design one 15" robot, utilize all of our research and development time to make a quality machine, and then clone it. The clone will be the 24" version. Our reasoning is simply due to a more streamlined design process and being able to focus resources better. In previous years, we have always had issues with two different robots. We believe this is the best way to go this year, especially given the rule analysis.


\subsection{Drivetrain Brainstorming}

\begin{table}[h!]
    \centering
    \begin{tabular}{lcccc}
        \toprule
        & Speed  & Torque & Simplicity & Strafing \\
        \midrule
        Tank    & 1 & 1   & 4 & 0 \\
        X        & 1.4 & 0.7 & 1 & 1 \\
        H        & 1   & 1   & 2 & 1 \\
        Mecanum  & 1   & 1   & 3 & 1 \\
        \bottomrule
    \end{tabular}
    \caption{Early design matrix for different drive trains.}
    \label{tab:drive design matrix}
\end{table}

Having identified the game challenge, the most important and overlooked part of a robot is the drivetrain. Considering the challenge of keeping the robots relatively compact and needing to effectively traverse the field, there are many drivetrains that would be highly compatible with this season's design challenge. The drivetrains considered are found in \ref{tab:drive design matrix} and the CAD depictions are found in \ref{fig:drivetrains}.

Note that the speed for the X drive is 1.4. This is because the drive can be viewed as two tank drives joined together at 90\degree. At which point, when both run independently, the velocity vectors of each would add together, resulting in a forward velocity of $\sqrt{2}$ times faster than tank with the same gearing and wheel diameter \cite{bib:drivetrainlore}. The speed of the other drive trains is mostly dependent on gearing and wheel diameter, and are assumed equivalent. Last is a tie between X and H which would contend with awkward wheel orientation and a perpendicular wheel respectively.


Torque scores are derived similarly to speed: assumed equivalent except for X with analogous reasoning. The scores for simplicity are ranks. Tank is clearly the most straightforward design, with no special concerns. Next is mecanum, where the main concern is each wheel being independently-powered (and the additional width). Then H which requires an additional wheel in the center of the robot. Then X which requires each wheel be independently powered and offset by 45\degree. Finally, bogie would require two additional center wheels, each wheel be independently-powered, and an additional linkage. Strafing is simply considered as a boolean value.


\section{Drivetrain design selection}
This year, our goal was to create a drive that can quickly move around the field whilst still having met necessary requirements to lay defensive. Considering the many drives listed in the initial design matrix \ref{tab:drive design matrix}. The most simple of the drives was the tank drive, scoring highly in simplicity.


Last season, the team ran tank drives on both bots. With tank being the most popular drive type and the most familiar drive type to the team, we believe that a tank drive offers the most simple yet effective solution for the 15" and 24". An X and H drive are simply too bulky for the 15". Due to the comfort and familiarity with the tank drive, the 15" design will be a tank drive.
\subsection{Problem Statement:}
Identify a desirable wheel size and gear ratio for the robot.
\subsubsection{Solution requirements:}
\begin{itemize}
    \item Must have at least 6 wheels with one traction wheel for stable defense.
    \item Must be faster than a direct-drive 200 RPM green cartridge.
    \item Must be powered by a 600 RPM cartridge.
\end{itemize}
\subsubsection{200 RPM vs 600 RPM:}
There are many combinations of wheel sizes and gear ratios. However, with the 15" robot, we want a drive that will be compact and balanced. Utilizing a 200 RPM green cartridge, we would need to have a speed ratio on the gear train. This is not ideal for a 6-motor drive. Additionally, green cartridges have more gear reduction internally, thus they have more friction. Blue cartridges have less of a gear reduction thus have less friction from there being less moving parts. Moreover, we will be able to gear the drive for torque, resulting in more compact gearing. This is ideal for a compact drive base.
\subsubsection{The Design:}
After initial testing from the first look of our gearing, which can be found in \ref{fig:Drivetrain Gearing}, it was discovered that having one traction wheel will provide a certain amount of friction to the field